% Created 2017-04-19 on. 18:49
\documentclass[11pt]{article}
\usepackage[utf8]{inputenc}
\usepackage[T1]{fontenc}
\usepackage{fixltx2e}
\usepackage{graphicx}
\usepackage{longtable}
\usepackage{float}
\usepackage{wrapfig}
\usepackage{rotating}
\usepackage[normalem]{ulem}
\usepackage{amsmath}
\usepackage{textcomp}
\usepackage{marvosym}
\usepackage{wasysym}
\usepackage{amssymb}
\usepackage{capt-of}
\usepackage{hyperref}
\tolerance=1000
\usepackage{minted}
\usepackage{tikz}
\usepackage{parskip}
\usepackage{color}
\usepackage{listings}
\usepackage{grffile}
\definecolor{mintedbackground}{rgb}{0.95,0.95,0.95}
\usepackage[natbib=true,uniquelist=false,bibstyle=authoryear-comp,citestyle=authoryear-comp,sorting=nyt,sortcase=false,sortcites=true,minbibnames=6,maxbibnames=6,maxcitenames=2,hyperref=false,backref=false,backend=bibtex,isbn=false,url=false,doi=false,eprint=false,firstinits=true,terseinits=true,dashed=false,uniquename=false,uniquelist=false]{biblatex}
\addbibresource{/home/alj/Dropbox.personal/Dropbox/Literature/CompleteLiterature.bib}
\usepackage[inline]{enumitem}
\usepackage{xcolor}
\hypersetup{
colorlinks,
linkcolor={red!50!black},
citecolor={blue!50!black},
urlcolor={blue!80!black}
}
\usepackage{tikz,graphics,graphicx}
\usetikzlibrary{decorations.shapes,arrows,decorations.pathreplacing,decorations.pathmorphing,backgrounds}
\usetikzlibrary{decorations.pathmorphing}
\usetikzlibrary{shapes.geometric}
\usepackage{setspace}%% The linestretch
\singlespacing
\usepackage[format=hang,indention=0cm,singlelinecheck=true,justification=raggedright,labelfont={normalsize,bf},textfont={normalsize}]{caption} %
\usepackage{vmargin}
\setpapersize{A4}
\setmarginsrb{2.5cm}{1cm}% links, oben
{2.5cm}{2cm}% rechts, unten
{12pt}{30pt}% Kopf: Höhe, Abstand
{12pt}{30pt}% Fuß: Höhe, AB
\usepackage{upquote}
%  use straight quotes when printing a command in minted
\AtBeginDocument{%
\def\PYZsq{\textquotesingle}%
}
\setlength{\parindent}{0pt}
\setlength{\parskip}{\baselineskip}
\definecolor{mintedbackground}{rgb}{0.95,0.95,0.95}
\author{Alexander Jueterbock, Martin Jakt\thanks{University of Nordland, Norway}}
\date{\textbf{PhD course: High throughput sequencing of non-model organisms}}
\title{\textbf{Recommended links and literature}}
\hypersetup{
 pdfkeywords={},
  pdfsubject={},
  pdfcreator={Emacs 24.5.1 (Org mode 8.3beta)}}
\begin{document}

\maketitle








The following sections list relevant literature, links to analysis
pipelines, as well as to tutorials and learning environments that get
you closer to become bioinformatics-experts.

\section{Recommended Reading for RAD sequencing}
\label{sec-1}

\begin{itemize}
\item \href{http://www.molecularecologist.com/2017/04/to-radseq-or-not-to-radseq/}{The Molecular Ecologist article: To RADseq or not to RADseq?}
\end{itemize}


\begin{itemize}
\item \href{https://www.nature.com/nrg/journal/v17/n2/full/nrg.2015.28.html}{Andrews, Kimberly R., et al. "Harnessing the power of RADseq for ecological and evolutionary genomics." Nature Reviews Genetics 17.2 (2016): 81-92.}
\end{itemize}


\begin{itemize}
\item \href{http://journals.plos.org/plosone/article?id=10.1371/journal.pone.0003376}{Baird, Nathan A., et al. "Rapid SNP discovery and genetic mapping using sequenced RAD markers." PloS one 3.10 (2008): e3376.}
\end{itemize}


\begin{itemize}
\item \href{http://www.nature.com/nmeth/journal/v9/n4/full/nmeth.1935.html}{Baker, Monya. "De novo genome assembly: what every biologist should know." Nature methods 9.4 (2012): 333-337}
\end{itemize}


\begin{itemize}
\item \href{http://onlinelibrary.wiley.com/doi/10.1111/mec.12354/abstract;jsessionid=259B878CB4F4CA43D108D850880842F7.f02t03?deniedAccessCustomisedMessage=&userIsAuthenticated=false}{Catchen, Julian, et al. "Stacks: an analysis tool set for population genomics." Molecular ecology 22.11 (2013): 3124-3140.}
\end{itemize}


\begin{itemize}
\item \href{http://onlinelibrary.wiley.com/doi/10.1111/1755-0998.12669/abstract}{Catchen, Julian M., et al. "Unbroken: RADseq remains a powerful tool for understanding the genetics of adaptation in natural populations." Molecular Ecology Resources (2017).}
\end{itemize}


\begin{itemize}
\item \href{http://journals.plos.org/plosone/article?id=10.1371/journal.pone.0106713}{DaCosta, Jeffrey M., and Michael D. Sorenson. "Amplification biases and consistent recovery of loci in a double-digest RAD-seq protocol." PloS one 9.9 (2014): e106713.}
\end{itemize}


\begin{itemize}
\item \href{http://bfg.oxfordjournals.org/content/9/5-6/416.short}{Davey, John W., and Mark L. Blaxter. "RADSeq: next-generation population genetics." Briefings in Functional Genomics 9.5-6 (2010): 416-423.}
\end{itemize}


\begin{itemize}
\item \href{http://journals.plos.org/plosgenetics/article?id=10.1371/journal.pgen.1000862}{Hohenlohe, Paul A., et al. "Population genomics of parallel adaptation in threespine stickleback using sequenced RAD tags." PLoS genetics 6.2 (2010): e1000862.}
\end{itemize}


\begin{itemize}
\item \href{http://onlinelibrary.wiley.com/doi/10.1111/j.1755-0998.2010.02967.x/abstract?deniedAccessCustomisedMessage=&userIsAuthenticated=false}{Hohenlohe, Paul A., et al. "Next‐generation RAD sequencing identifies thousands of SNPs for assessing hybridization between rainbow and westslope cutthroat trout." Molecular ecology resources 11.s1 (2011): 117-122.}
\end{itemize}


\begin{itemize}
\item \href{http://onlinelibrary.wiley.com/doi/10.1111/1755-0998.12635/abstract}{Lowry, David B., et al. "Breaking RAD: an evaluation of the utility of restriction site‐associated DNA sequencing for genome scans of adaptation." Molecular ecology resources 17.2 (2017): 142-152.}
\end{itemize}


\begin{itemize}
\item \href{http://onlinelibrary.wiley.com/doi/10.1111/1755-0998.12677/abstract}{Lowry, David B., et al. "Responsible RAD: Striving for best practices in population genomic studies of adaptation." Molecular Ecology Resources (2017).}
\end{itemize}


\begin{itemize}
\item \href{http://onlinelibrary.wiley.com/doi/10.1111/1755-0998.12649/abstract}{McKinney, Garrett J., et al. "RADseq provides unprecedented insights into molecular ecology and evolutionary genetics: comment on Breaking RAD by Lowry et al.(2016)." Molecular Ecology Resources (2017).}
\end{itemize}


\begin{itemize}
\item \href{http://journals.plos.org/plosone/article?id=10.1371/journal.pone.0037135#pone-0037135-g005}{Peterson, Brant K., et al. "Double digest RADseq: an inexpensive method for de novo SNP discovery and genotyping in model and non-model species." PloS one 7.5 (2012): e37135.}
\end{itemize}


\begin{itemize}
\item \href{http://www.biomedcentral.com/1471-2164/15/275}{Rasic, Gordana, et al. "Genome-wide SNPs lead to strong signals of geographic structure and relatedness patterns in the major arbovirus vector, Aedes aegypti." BMC genomics 15.1 (2014): 275.}
\end{itemize}


\begin{itemize}
\item \href{http://www.biolbull.org/content/227/2/146.short}{Schweyen, Hannah, Andrey Rozenberg, and Florian Leese. "Detection and Removal of PCR Duplicates in Population Genomic ddRAD Studies by Addition of a Degenerate Base Region (DBR) in Sequencing Adapters." The Biological Bulletin 227.2 (2014): 146-160.}
\end{itemize}


\begin{itemize}
\item \href{https://peerj.com/articles/431/}{Puritz, Jonathan B., Christopher M. Hollenbeck, and John R. Gold. "dDocent: a RADseq, variant-calling pipeline designed for population genomics of non-model organisms." PeerJ 2 (2014): e431.}
\end{itemize}


\begin{itemize}
\item \href{http://onlinelibrary.wiley.com/doi/10.1111/mec.12965/full}{Puritz, Jonathan B., et al. "Demystifying the RAD fad." Molecular ecology 23.24 (2014): 5937-5942.}
\end{itemize}


\begin{itemize}
\item \href{http://ngs-expert.com/tag/rad-seq/}{Blog on RAD seq}
\end{itemize}

\section{Guidelines for pooled sequencing data}
\label{sec-2}
\begin{itemize}
\item \href{http://www.nature.com/nrg/journal/v15/n11/full/nrg3803.html}{Schlötterer, Christian, et al. "Sequencing pools of individuals - mining genome-wide polymorphism data without big funding." Nature Reviews Genetics (2014).}
\end{itemize}

\section{Useful programs and analysis pipelines}
\label{sec-3}

\begin{itemize}
\item \href{http://onlinelibrary.wiley.com/doi/10.1111/men.2017.17.issue-1/issuetoc}{\emph{Molecular Ecology Resources} Jan 2017; Special Issue: Population Genomics with R}

\item \href{http://code.google.com/p/popoolation/}{Popoolation}: Population genomic analysis of pooled samples, see also \href{http://drrobertkofler.wikispaces.com/file/view/pooledAnalysis_part1.pdf/489488280/pooledAnalysis_part1.pdf}{this presentation}
\end{itemize}


\begin{itemize}
\item \href{http://code.google.com/p/popoolation2/}{Popoolation 2}: allows comparison of allele frequencies between two or more populaitons
\end{itemize}


\begin{itemize}
\item \href{http://sfg.stanford.edu/guide.html}{The Simple Fool's Guide to Population Genomics via RNA-Seq}
\end{itemize}


\begin{itemize}
\item \href{http://www.bioconductor.org/}{Bioconductor}: R packages for genomic data analysis
\end{itemize}


\begin{itemize}
\item \href{http://rosalind.info/problems/locations/}{Rosalind}: Learning python
\end{itemize}


\begin{itemize}
\item \href{http://biopython.org/wiki/Main_Page}{Biopython}
\end{itemize}


\begin{itemize}
\item \href{http://www.bioperl.org/wiki/Main_Page}{BioPerl}
\end{itemize}


\begin{itemize}
\item \href{http://creskolab.uoregon.edu/stacks/}{STACKS}: building loci from short sequences and analyzing RADseq data
\end{itemize}


\begin{itemize}
\item \href{https://ddocent.wordpress.com/ddocent-pipeline-user-guide/}{Ddocent}: ddRAD analysis pipeline
\end{itemize}

\section{Recommended books}
\label{sec-4}
\begin{itemize}
\item \href{http://unixandperl.com/}{Unix and Perl to the Rescue}
\item \href{http://www.staff.hs-mittweida.de/~wuenschi/doku.php?id=rwbook2}{Computational Biology}
\item \href{https://www.amazon.com/Primer-Analysis-Genomic-Data-Using/dp/331914474X/ref=sr_1_1?ie=UTF8&qid=1491488356&sr=8-1&keywords=primer+to+analysis+of+genomic+data+using+r}{Primer to Analysis of Genomic Data Using R (Use R!)}
\item \href{https://www.amazon.com/Bioinformatics-Data-Skills-Reproducible-Research/dp/1449367372/ref=sr_1_1?ie=UTF8&qid=1491488394&sr=8-1&keywords=bioinformatics+data+skills}{Bioinformatics Data Skills: Reproducible and Robust Research with Open Source Tools}
\item \href{https://www.amazon.com/Practical-Computing-Biologists-Steven-Haddock/dp/0878933913/ref=sr_1_8?ie=UTF8&qid=1491813979&sr=8-8&keywords=computational+biology}{Practical Computing for Biologists}
\end{itemize}
\section{Upcoming \href{https://www.coursera.org/}{Coursera courses}}
\label{sec-5}
\begin{itemize}
\item \href{https://www.coursera.org/course/rprog}{R programming}
\end{itemize}


\begin{itemize}
\item \href{https://www.coursera.org/course/algobioprogramming}{Algorithms, Biology, and Programming for Beginners}
\end{itemize}


\begin{itemize}
\item \href{https://www.coursera.org/course/epigenetics}{Epigenetic Control of Gene Expression}
\end{itemize}


\begin{itemize}
\item \href{https://www.coursera.org/course/genbioconductor}{Bioconductor for Genomic Data Science}
\end{itemize}


\begin{itemize}
\item \href{https://www.coursera.org/course/genstats}{Statistics for Genomic Data Science}
\end{itemize}


\begin{itemize}
\item \href{https://www.coursera.org/course/comparinggenomes}{Comparing Genes, Proteins, and Genomes (Bioinformatics III)}
\end{itemize}


\begin{itemize}
\item \href{https://www.coursera.org/course/genpython}{Python for Genomic Data Science}
\end{itemize}


\begin{itemize}
\item \href{https://www.coursera.org/course/gencommand}{Command Line Tools for Genomic Data Science}
\end{itemize}
Emacs 24.5.1 (Org mode 8.3beta)
\end{document}